% \documentclass[10pt,a4paper,twocolumn]{article}
\documentclass[10pt,a4paper]{article}
\usepackage[utf8]{inputenc}
\usepackage[T1]{fontenc}
%\usepackage{gentium}
\usepackage[labelfont=bf]{caption}
\usepackage{float}
\usepackage{mathptmx} % Use Times Font
\usepackage{amsmath}
\usepackage{tabularx}
\usepackage{amsfonts}
\usepackage{graphicx} % Required for including pictures
\usepackage{hyperref} % Format links for pdf
\usepackage{biblatex}
\addbibresource{references.bib}
\usepackage{booktabs} % Used so that tables generated by pandas
                      % to_latex() work correctly
\usepackage{multicol}

\usepackage[symbol]{footmisc}
\renewcommand{\thefootnote}{\fnsymbol{footnote}}

\frenchspacing % No double spacing between sentences
\usepackage[margin=1in]{geometry}

\usepackage[all]{nowidow} % Tries to remove widows
\usepackage[protrusion=true,expansion=true]{microtype} % Improves typography, load after fontpackage is selected

\usepackage{lipsum} % Used for inserting dummy 'Lorem ipsum' text into the template

\setlength{\parindent}{7pt}

\title{Investigating the Significance of Elo in Online Chess Games}
\author{Leon Lee and Lila Marshman}

\begin{document}

\maketitle

%% INSTRUCTIONS:
%%
%% 1. Create your own copy of this Overleaf project. You can either edit your report
%% using:
%%
%%    a. Overleaf professional, a collaborative LaTeX editor. You can click
%%       "Copy Project" from the Overleaf menu to create a version where you have
%%       read and write permissions. See the following for documentation:
%%       https://www.overleaf.com/edu/edinburgh and
%%       https://uoe.sharepoint.com/:f:/r/sites/digitalskillsandtraining/Shared%20Documents/LaTeX/LaTeX%20for%20Beginners%20using%20Overleaf?csf=1&web=1&e=cPqTI3
%%
%%    b. A LaTeX editor on your PC. For this option, you can download the source
%%       of this project as a zip (via the Overleaf menu).
%% 
%% 2. Please rename this file fds-project-option-1.tex, 
%% fds-project-option-2.tex, or fds-project-option-3.tex, depending on
%% which project option you are doing. When you submit, please submit
%% the PDF file with the corresponding name.
%% 
%% 3. Please keep the section and paragraph headings as they
%%    are. You should delete all the text within the headings, e.g.
%%    the text that says "What is the area of this data science
%%    study, and why is it interesting to investigate" and the
%%    bullet points. Keeping the headings makes the report a lot
%%    easier for the markers to read, and making things easy for
%%    markers is always beneficial.
%%
%% 4. The word limit for the Overview section is mandatory. For the
%% other sections word limits are suggested.
%%
%% 5. The page limits must be strictly adhered to, and depend on if
%% you are working individually, in pairs or in threes:
%%
%%   - Individual: 6 pages 
%%   - Pairs: 8 pages 
%%   - Threes: 10 pages 
%%
\section{Overview}
% 250 words maximum
Online chess sites receive a huge amount of game data from large volumes of users playing on their sites. Many have large player bases, representing a huge variety in player Elo scores (a number corresponding to skill level). We used data from games played on the online chess website Chess.com to analyse the relevancy of players' Elo in a chess game. A logistic regression model was used to investigate the effect of the difference in Elo on winning a game. Interpreting the regression coefficients provided insight into the strength of this correlation for different Elo classes, and our results were evaluated using a Wald test. \newline

Following this, by using an exponential function "Temptation", we visualised a player's likelihood of playing a specific chess tactic called \textit{en passant} across three different categories of Elo. We then attempted to predict a player's Elo using the context of this \textit{en passant} move. By analysing computer-generated evaluations of the state of the game preceding the move, and by using Temptation as a hyperparameter for a predictive model, we attempted to utilise PCA and a k-Nearest Neighbour (k-NN) algorithm to classify users into three classes of Elo. To evaluate the model's predictive performance we calculated various metrics (accuracy, precision, recall, F1 score). Despite our investigation identifying a correlation between Elo difference and winning, we did not find sufficient evidence that Elo was predictable from the context of an \textit{en passant} move. \newline

All decimal values in this report are provided to 4 decimal places.


\begin{multicols}{2}
[
\section{Introduction}
]
% Suggested 400 words

\paragraph{Context and motivation}

Online chess sites such as Chess.com allow users to play with friends or strangers, offering a wide variety of chess variants and time controls to play with. Online chess' rise in popularity follows the increase in free time during the COVID-19 pandemic lockdowns, the popularity of Netflix's show 'The Queen's Gambit', and world-ranking players streaming the game on Twitch \cite{The2020ChessBoom} With a sudden increase in online players comes an increase in publicly available game data - this provides a perfect opportunity for an investigation into player's skill (measured by Elo points). In this study, we explore the impact of a player's skill level on a game's direction and patterns in the play styles of players at different Elos. \newline

The particular play style we explore in this study is the context surrounding an \textit{en passant} chess move. Typically, chess tactics are something you aim for, therefore there is the natural assumption that a higher-ranked player would be able to set up certain tactics more consistently than a lower Elo one. One exception to this is the move \textit{En passant}. It is an incredibly situational move, and it heavily relies on a player's opponent to move a certain way for a player to be able to play it in the first place. In the chess scene, particularly in the online chess community, \textit{en passant} has gained a cult-like following\cite{EnPassant}. It's become a popular running joke amongst players to always capture via \textit{en passant} when given the chance, even if this puts them in a worse position than having not chosen that move. Losing an online game results in your Elo rating decreasing, thus most would expect highly rated players to not risk a bad game position, hence not capture \textit{en passant} unless it's beneficial. In this study we explore whether we can use the context surrounding an \textit{en passant}-allowed board state to predict whether a player falls into the low, medium, or high Elo category. \newline

Insight into these areas, particularly if the second investigation proves Elo is predictable from move contexts, may be useful in determining how players in the past compare to today's players. There is much speculation on how historical chess champion Bobby Fischer would compare to current world champion Magnus Carlsen \cite{BobbyFischerVsMagnusCarlsen}, thus if we are able to predict a modern-day Elo for Fischer by inputting his play style information into a model trained on modern games, we may find evidence suggesting how he'd compare.

\paragraph{Previous work}

A number of works in the literature have found a correlation between Elo disparity and winning a chess game. One book proposes a correlation between comparative Elo and winning, whereby the chance of winning against a higher-rated player decreases with a larger Elo difference \cite{PsychologyOfChessSkill}. An article published on the website 'Towards Data Science' also found a correlation between Elo disparity and winning, but only within a range of $\pm 50$ Elo points difference \cite{HowMuchDoesEloMatter}. They found that outside of $\pm 50$ Elo points, the probability of a win did not increase or decrease significantly with a further increase or decrease in Elo \cite{HowMuchDoesEloMatter}. \newline

Some existing studies investigate and design tests to predict chess proficiency. One paper devises the 'Amsterdam Chess Test', a 5-task chess test which reliably predicts Elo based on a person's answers during each of these 5 chess-related tasks \cite{PsychometricAnalysisChessExpertise}. Another paper suggests eye movements of participants when faced with on-screen chess questions are a successful predictor for Elo \cite{VisualPerceptionRankingChess}. However, there appears to be no literature directly studying the association between a player's response to specific game moves and Elo, and none specifically investigating the context of \textit{en passant} moves.


\paragraph{Objectives}

Our goals in this study are to investigate whether there is a correlation between a player's Elo, and the moves they play in a game. In our case we are focusing on the move \textit{en passant}, and a player's willingness to play the move if an opportunity presents itself. We will first take a bird's eye view of the overall effects of Elo, and investigate whether a significant gap in Elo affects the chances of winning. Then, we will analyse games containing the move \textit{en passant}, and through the context of the board state try and determine a player's Elo, and indirectly, their chance of winning.
% What questions are you setting out to answer?

\section{Data}
% Suggested 300 words

% Who created the dataset(s)?  How you have
% obtained it (e.g., file or web scraping), and do the T\&Cs allow you
% to use obtain the data for the project?

\paragraph{Data provenance}
We obtained our data from Kaggle.com \cite{Kaggle}, where they provided a dataset of over 60,000 games of chess taken from Chess.com. The User Agreement on Chess.com states that you are not allowed to data mine \cite{ChessT&C}, but in this case the dataset was extracted using the Chess.com API so it complies with their regulations.


\paragraph{Data description}
The dataset records $66,879$ games of chess that took place on Chess.com with varying game modes, time classes, and levels of players. Information about each player is provided, i.e. the usernames, Chess.com profiles, and Elo rating during the match. Information about the game is also provided, i.e. the result, information about the time rules, whether it was rated, and the final chess board in a notation called "FEN". The final column is called the PGN (Portable Game Notation) - a column in a standard format to be easily read by other chess analysers. Within the PGN, there includes a list of the moves that took place during the game. Using this we can simulate and replay exactly how the game was played, and provide further analysis.


\paragraph{Data processing}
For both investigation questions, we removed alternate game modes that were not standard chess (for example, 'Chess960', and all game records if they contained any NaN values. We removed games where the game terminated early by using two filters: determining no major pieces had been moved (by reading the top row of the 'FEN', and removing games with less than 10 total moves (5 per player). We believed any games in these two categories would not be useful to the data analysis. Additionally, we removed games from the 'daily' time class. This was due to these records' 'PGN' column being structured in a different format, meaning daily games would have needed to be analysed separately from other modes. Unlike the rest of the remaining games, 'daily' games had no time pressure on players - thus removing these removes the chance that time pressure may have affected the results as a confounding variable. Due to this, and the fact there was only a small amount of 'daily' games ($\sim9.2\%$ of total games), we removed these from the dataset.\newline

For the first investigation, we required a binary outcome (winning or losing) in order to use logistic regression, thus we additionally excluded all games which ended in a draw or other indeterminate end-states. \newline

For our second investigation, we utilised various external libraries to assist in defining k-NN hyperparameters. Using the \textit{python-chess} library \cite{python-chess}, we parsed the "PGN" field to filter out games that didn't include an \textit{en passant} opportunity ($n=5,074$), and highlight games where an \textit{en passant} move actually occurred ($n=1,563$). We then used \textit{Stockfish}\cite{StockFish}, an open-source chess game engine, to evaluate how much of an advantage from an initial board state a player would gain. Advantage is counted in centipawns (cp), and is always positive from White point of view. Evaluations were calculated from the following moves: a Stockfish calculated best move, the move the player decided to make, and finally the relevant \textit{en passant} capture. These scores were added to the dataframe. All non-numerical data were converted to numerical data by assigning numerical categories. This was important for standardising the data, performing a PCA and creating the k-NN model. It was also important for the k-NN model that all sample data was independent, therefore we removed games where a player's username already exists in the sample.\newline


\begin{figure*}[p]
  \centering
  \includegraphics[width=\textwidth]{report/images/log_regression_dual.png}
  \caption{Subplot 1: Each datapoint represents how many Elo points a player is than their opponent during a game, and whether they won (1) or lost (0). The standard logistic function is plotted in red. 'Elo mismatch size' is a measure of the number of Elo points a player's opponent is above them. \newline \newline  
  Subplot 2: Each line represents the logistic function plotted for a particular Elo difference class. Information on the odds ratios for each is provided, calculated from $e^{\beta_{1}}$ for each class' regression coefficient $\beta_{1}$. The red line shows the overall standard logistic function from subplot 1, to aid slope comparisons for each class.}
  \label{fds-project-template:fig:log_regression}
\end{figure*}



\section{Exploration and  analysis}
% Suggested 500 words for individual report; proportionately longer
% for group projects).

% 't' means "try to position at the top of the page"

% 'b' means "try to position at the bottom of the page"

\paragraph{Data Analysis: Question 1 - Investigating the relationship between players' Elo difference and winning}

We used logistic regression to investigate the association between the difference in players' Elo and winning. We believed this to be the most appropriate technique to use since logistic regression typically works well for data with a continuous predictor (Elo difference) and a binary response variable (winning or losing). The differences in Elos for each game were calculated from the perspective of a white-playing player. For example, if in a game, white had and Elo of $1,000$ and black had an Elo of $900$, the difference in Elo for this game would be recorded as $-100$. \newline

After applying logistic regression to the sample data we visualised the results (Figure \ref{fds-project-template:fig:log_regression}) and found the regression coefficients $\beta_{0}$ and $\beta_{1}$ to be $0.0768$ and $0.0108$ respectively. $\beta_{0}$ describes the log odds for opponents of the same Elo. Since it is close to $0$ ($0.0768$ logits), this tells us winning or losing are almost equally likely for players with $0$ difference in Elo. This is shown in Figure \ref{fds-project-template:fig:log_regression}, Subplot 1 where we see the logistic function has an almost $0.5$ win probability where a player's opponent is $0$ Elo points higher than them. We used $\beta_{0}$ to calculate this probability exactly as 
$$\displaystyle\frac{1}{1+e^{\beta_{0}}} = 0.48081.$$
Furthermore, the odds ratio $e^{\beta_{1}} = 1.0108$ shows that for every 1 point higher a player is in Elo, they are $1.0108$ times more likely to win against their opponent. Figure \ref{fds-project-template:fig:log_regression}, Subplot 1's logistic function line also shows that at around $\pm 400$ Elo points difference, the outcome is predicted as almost certainly a win for the player with a higher Elo. Figure \ref{fds-project-template:fig:log_regression}, Subplot 1 also shows that in the sample used, white-playing players against opponents rated $1,500$ Elo points lower than them always won. Similarly, white-playing players against opponents rated $1,200$ Elo points above them always lost. \newline

Figure \ref{fds-project-template:fig:log_regression} Subplot 2 shows the different logistic functions produced for classes of differing Elo, alongside the main model's overall logistic function. The trend for the regression coefficients $\beta_{1}$ typically show that the larger the Elo difference, the smaller the odds ratio $\beta_{1}$ is. Thus for larger differences in Elo, the rate of increase in likelihood of a player beating their opponent decreases, despite the actual likelihood of beating their opponent increasing. \newline

The logistic regression model was estimated using maximum likelihood, so it seemed most appropriate to use a Wald test to test for a relationship between Elo difference and the probability of a win \cite{WaldTest}. We used the null hypothesis $H_{0}: \beta_{1} = 0$ which states there is no statistically significant relationship, and alternative hypothesis $H_{a}: \beta_{1} \neq 0$ which suggests there is a statistically significant relationship. Following the test procedure outlined by Forthofer, Lee and Hernandez \cite{WaldTest}, and a method to calculate the Wald statistic inspired by StackExchange user j\_sack \cite{StackExchangeWaldTest}, we found a Wald statistic of $75.3946$, which gave a p-value of $p<0.01$. Thus we may reject the null hypothesis at the $1\%$ level, concluding there is sufficient evidence to reject the notion that there is not a statistically significant relationship between Elo difference and the probability of winning. \newline

\paragraph{Data Analysis: Question 2 - Using the context of an en-passant move to attempt to predict Elo of a player}

A small number of factors can be used as context for the \textit{en passant} move. We believed our model would have the highest predicting power given the most information, thus defined the following hyperparameters to use in our k-NN model:

\begin{itemize}
\addtolength\itemsep{-2.5mm}
  \item Player's colour
  \item Did the player choose the \textit{en passant} move
  \item Was the game rated
  \item Game time class
  \item Time taken to make the potential \textit{en passant} move
  \item Stockfish evaluation of the player's actual move
  \item Temptation score
  \item Probability that a similar grouped player will play an en-passant move
\end{itemize}

Since player Elos were normally distributed, we used the sample mean ($\bar{X} = 1,240$) and standard deviation ($\sigma = 400$) to derive $3$ Elo classes such that the amount of data classed in each low, medium and high Elo group was in an almost $30:40:30$ ratio. We believed this ratio most effectively balanced not allowing the upper bound of a low Elo too close to the lower bound of a high Elo (thus the 'low', 'medium' and 'high' Elo classes make somewhat intuitive sense) whilst ensuring there wasn't a huge difference in the amount of data categorised in each class. Thus the resulting class boundaries became: $0 - 1000$ Elo, $1001 - 1400$ Elo, and greater than $1401$ Elo. The sample's distribution of each class on different levels of temptation is visualised in Fig. \ref{fds-project-template:fig:ep_distplot}. We defined the function "temptation" to estimate how viable a player would find an \textit{en passant} move, modelled as such:
$$T(\text{move})= 1 -e^{(E_{\text{best}} - E_{\text{move}})/{n}}$$
This is a very basic model but it ultimately works off the basis that for a player making a move, a very badly evaluated move would be significantly less likely to be chosen than a move close in evaluation to the best move. The number $n$ is a factor to stop the exponential function from growing too fast, and no significant value of $n$ proved to be significantly better than others so we chose $n=220$. Another approach to finding the "temptation" of a move is shown in Oshri, B., \& Khandwala, N. \cite{mcilroy2022learning}.\newline

\begin{figure*}[t]
  \centering
  \includegraphics[width=\textwidth]{report/images/ep_distplot.png}
  \caption{Every value of temptation on the chart is negative since we assume a move cannot be better than the Stockfish evaluated best move. Values with a temptation of $0$ were excluded as the number of games where en-passant did occur if it was the best move was overwhelmingly high across every class.}
  \label{fds-project-template:fig:ep_distplot}
\end{figure*}

After plotting this graph, we tried to emulate these results by creating a new field for the probability that someone in a certain Elo would play an en-passant move, named "Quotient". There weren't enough samples in the dataset to get any meaningful information from comparing single values, so we rounded each player's Elo to the nearest $10$ and calculated the quotient using the following formula\footnote[1]{Note: this method uses the Elo to find the quotient which means it's flawed as you cannot 100\% predict Elo using this, see Section 5.}
$$\mathbb{P}(\text{Elo class}) = \frac{\text{Games with successful e.p. captures}}{\text{Total games with e.p. opportunity}}$$

Once we calculated all our variables, we then split our dataset in a $60:40$ ratio for the training and test set respectively, stratifying by Elo class to reduce the impact of the slight imbalanced class distribution (Fig. \ref{fds-project-template:fig:knn}). Using the training set, we performed a PCA analysis to reduce the dimensionality of the dataset, and then utilised a k-nearest neighbours algorithm to classify the test set.\newline

From Fig. \ref{fds-project-template:fig:ep_distplot}, we can see that higher Elo players have a very concentrated area of temptation where they will perform a capture, i.e. when the temptation is close to $0$. The drop-off after being relatively sharp, and turning to a near-zero chance for any move below $T(move)=-20$. On the other hand, both the medium and lower Elo classes have a much spread-out and shallower slope, which indicates that in lower Elos, the novelty of playing en-passant does outweigh the potential repercussions of playing a bad move. However, all three Elo classes seem to follow a normal distribution, implying it would be possible to predict how a player at a certain ELO will play given an en-passant opportunity.\newline

On the other hand, from the k-NN Chart (Fig. \ref{fds-project-template:fig:knn}), the left and right sides are somewhat defined, but the middle Elo class has a very erratic classifying area, with it being blended in with the other two classes as well. This is likely due to there being a lot of overlap from all three categories in the section where most medium Elo games lie, making it difficult to accurately classify a data point inside of that region from the parameters that we used. From these two graphs, it is clear that although we can somewhat predict the chances of an en-passant move given the Elo of a player, the inverse is not true and using only context surrounding an \textit{en passant} move is not nearly enough to predict someone's Elo without a much more accurate model and further parameters of the players involved. \newline

To further justify this, we created a confusion matrix (Table \ref{tab:confusion_matrix}) detailing the numbers of correct and incorrect predictions of the k-NN model on the test set for each Elo class. 
\begin{figure*}[t]
  \centering
  \includegraphics[width=\textwidth]{report/images/knn_graph.png}
  \caption{Each dot in the chart represents a unique player in the dataset, and their Elo is sorted into the corresponding category of Elo class. The algorithm is tuned with between 3 and 7 neighbours and picks the result with the highest accuracy, in this case 5.}
  \label{fds-project-template:fig:knn}
\end{figure*}

\begin{table}[H]
  \centering
  \caption{Confusion matrix of the test set, providing information on the number number of data points the model predicted to be in each Elo class, alongside the actual amount in each Elo class. Here Elo classes are coded as: $1: 0 - 1,000, \qquad 2: 1,001 - 1,400, \qquad 3: 1,401+$}
  \label{tab:confusion_matrix}
    \begin{tabular}{lrrr}
        \toprule
        &\textbf{Predict 1}&\textbf{Predict 2}&\textbf{Predict 3}\\
        \midrule
        \textbf{Actual 1}&$82$&$ 75$&$70$\tabularnewline
        \textbf{Actual 2}&$17$&$ 35$&$39$\tabularnewline
        \textbf{Actual 3}&$ 99$&$ 121$&$95$\tabularnewline
        \bottomrule
    \end{tabular}
\end{table}

Values from the confusion matrix were used to calculate metrics to evaluate the model's predictive performance. These metrics were: accuracy ($0.3349$), precision ($0.4020$), recall ($0.3349$), and sample-weighted F1 score ($0.3518$). The accuracy is below $0.5$, thus demonstrating our model has poor predictive capabilities. Furthermore, it's very close to $\frac{1}{3}$, suggesting the k-NN model is no better at predicting Elo than if it were randomly guessing the Elo category on a uniform distribution for each data piece. The precision and recall scores were used in the calculation for the F1 score, which evaluates the model for precision (amount of correctly classified data points) and robustness (whether it misses a significant amount of data) \cite{MetricsToEvaluateYourML}. To account for the test set containing different proportions of each Elo class, we used the sample-weighted calculation for the F1 score. This produced an F1 score of $0.3518$ which shows a bad fit of the model to the data, implying Elo cannot be accurately predicted from the model's hyperparameters. \newline

% \begin{figure*}[t]
%   \centering
%   \includegraphics[width=\textwidth]{report/images/temptation_chart.png}
%   \caption{descriptive caption that i cba doing rn}
%   \label{fds-project-template:fig:temptation}
% \end{figure*}



\section{Discussion and conclusions}
% Suggested 400 words.

\paragraph{Summary of findings}
This study's findings indicate that the amount of difference in Elo between two players affects the outcome of a chess game, for games played on the online chess website Chess.com.  \newline

By focusing on the situational board state whenever an \textit{en passant} chess move was played, we developed a k-NN model to determine whether there was enough information from this context to predict the Elo of the player. However we
found insufficient evidence that a player's Elo was predictable from the context of this one move.


\paragraph{Evaluation of own work: strengths and limitations}
A caveat of using Stockfish is that since it is a live engine, without utilising significant processing power, which takes increasingly longer amounts of time the stronger you make the engine, it will end up with slightly different results each time it runs. Although we have found that in most cases the data is similar, sometimes it generates a large variation in the graphs that will be shown in this study.\newline

Another flaw is that while the "quotient"  field measuring the probability of a player playing an \textit{en passant} move cannot be used backwards to predict the Elo of the player, the derivation of the value is taken from rounding the player's Elo. So this study is more of a proof of concept, as it cannot be viable to predict a person's Elo blind. A more viable option is to instead of grouping by similar Elo class, group by individual players instead. However, in this dataset, there is not a large enough sample to viably do this, as the mean number of games per player is only $1.7370$, with a standard deviation of $4.0367$.


\paragraph{Comparison with any other related work}
Our study supports the results both Holding \cite{PsychologyOfChessSkill} and Onofrey \cite{HowMuchDoesEloMatter} received, who also found a correlation between Elo disparity and winning. However whilst Onofrey \cite{HowMuchDoesEloMatter} determined a relationship to be most prominent within $\pm 50$ Elo points difference, we determined this between roughly $\pm 400$ points difference. We also further found that as Elo gap increased, the odds ratio appeared to decrease, which perhaps provides the answer to why Onofrey found that outside of $\pm 50$ Elo points, the
probability of a win will not change significantly with a further change in Elo \cite{HowMuchDoesEloMatter}. \newline

Our way of calculating temptation is similar to Oshri, B., \& Khandwala, N. \cite{mcilroy2022learning}, where they trained a chess game engine "Maia" to act as a human would. It states \textit{"We measure "move quality" by using Stockfish depth 15's evaluation function, then converting its evaluation into a probability of winning the game"}. However, instead of comparing a move to the best move, they obtained their version of a temptation by comparing it against a ratio of winning players and the total number of observations at that evaluation value.



\paragraph{Improvements and extensions}

The model used to predict the "temptation" of a move is very simplistic and does not consider other parameters other than the evaluations given by Stockfish. In the real world, there would be lots of other factors such as remaining time, or the fact that a human would not evaluate moves in the same way as a computer. Thus tuning the hyperparameters would improve our k-NN model. In Fig. \ref{fds-project-template:fig:ep_distplot}, the results that were displayed were what we were expecting in this study. If improved and tuned better, the idea for the model could be used to analyse moves beyond just \textit{en passant}.

\end{multicols}

\printbibliography
\end{document}
